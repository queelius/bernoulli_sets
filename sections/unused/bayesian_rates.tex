%%%%%%%%%%%%%%%%%%%%%%%%%%%%%%%%%%
% Bayesian treatment of uncertain rates
% Extracted from uncertain_rates.tex — joint density framework
% for future extension.
%%%%%%%%%%%%%%%%%%%%%%%%%%%%%%%%%%

% Original context: This content appeared at the beginning of
% sections/uncertain_rates.tex and develops a Bayesian marginal
% integration approach to uncertain rates. It was removed because
% the section's primary contribution is the interval arithmetic
% approach, and the Bayesian framework was disconnected from the
% rest of the exposition.

We may not be certain about the \emph{expected} false positive and true positive rates, i.e., we may only have the joint distribution of $\tilde{A}$, $\alpha$, and $\beta$.

A relatively easy case to analyze is the positive-negative (second-order) random approximate set model.
Suppose we are interested in the distribution of $\tilde{A}$ given $R$ has $p$ positives and $n$ negatives.
Since we are primarily interested in the distribution of false positives and true positives (or their corresponding rates), we consider the related random tuple $\Tuple{\FP_n, \TP_p, \TPR, \alpha}$ which, assuming $\alpha$ and $\TPR$ are independent, has a joint probability density function given by
\begin{equation}
	f_{\TP_p, \FP_n,\TPR,\alpha}(t,f,\tprate,\fprate) = f_{\TP_p,\TPR}(t, \tprate) f_{\FP_n,\alpha}(f,\fprate)
\end{equation}
where
\begin{align}
	f_{\TP_p,\TPR}(t,\tprate) &= f_{\TP_p | \TPR}(t | \tprate) f_{\TPR}(\tprate),\\
	f_{\FP_n,\alpha}(f,\fprate) &= f_{\FP_n | \alpha}(f | \fprate) f_{\alpha}(\fprate).
\end{align}

When we \emph{marginalize} over the true positives, we get the result
\begin{equation}
\begin{split}
	f_{\TP_p}(t) &= \int_{0}^{1} f_{\TP_p | \TPR}(t | \tprate) f_{\TPR}(\tprate) \dif \tprate\\
	             &= \int_{0}^{1} \binom{p}{t} \tprate^{t}(1-\tprate)^{p-t} f_{\TPR}(\tprate) \dif \tprate.
\end{split}
\end{equation}

If all the probability mass for $\TPR$ is assigned to a particular point $\tprate$, the probability mass function simplifies to
\begin{equation}
    f_{\TP_p}(t) = \binom{p}{t} \tprate^{t}(1-\tprate)^{p-t},
\end{equation}
which is probability mass function of a binomial distribution.

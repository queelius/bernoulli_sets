%%%%%%%%%%%%%%%%%%%%%%%%%%%%%%%%%
% Bernoulli model
%%%%%%%%%%%%%%%%%%%%%%%%%%%%%%%%%

\section{Bernoulli set model}
In the \emph{Bernoulli} set model, we describe the statistical properties of processes that \emph{generate} approximations of a certain kind that model many existing processes and generalizes to higher-order approximations under algebraic composition.

In what follows, we specify the axioms of the Bernoulli set model.

Theoretically, a process that generates approximations could exhibit correlations of any sort, but Bernoulli sets are constrained by the following axiom.
\begin{axiom}[Element-wise independence]
\label{asm:element_indep}
For all distinct $x, y \in U$, the error events at $x$ and $y$ are mutually independent:
\begin{equation}
P(\mathbf{1}_{\tilde{A}}(x) \neq \mathbf{1}_{A}(x) \mid \mathbf{1}_{\tilde{A}}(y) \neq \mathbf{1}_{A}(y)) =
	P(\mathbf{1}_{\tilde{A}}(x) \neq \mathbf{1}_{A}(x)).
\end{equation}
More generally, the family $\{\mathbf{1}_{\tilde{A}}(x) : x \in U\}$ is mutually independent given the objective set $A$ and the error rates.
\end{axiom}

The complexity in the Bernoulli set model stems from the fact that different subsets of the universal set may exhibit different error rates.
Suppose the objective set $R \in 2^U$ induces a partition $\{B_1, \ldots, B_n\}$ of $U$ such that elements within each block $B_i$ share a common error rate $\alpha_i$.
In the first-order (positive-negative) model, $n = 2$: the partition is $\{R, R^c\}$ with block error rates $\alpha_1 = \beta$ (false negative rate on positives) and $\alpha_2 = \alpha$ (false positive rate on negatives).
\begin{axiom}[Conditional independence of block error rates]
\label{asm:fpr_fnr_r_indep}
Given the objective set $R$, the random error rates $\alpha_1, \ldots, \alpha_n$ for the respective partition blocks $B_1, \ldots, B_n$ are mutually independent.
\end{axiom}
For instance, in the first-order model, the random false positive rate $\alpha$ and the random false negative rate $\beta$ are conditionally independent given $R$.

There are two natural characteristics of the Bernoulli set model, the random false positive and false negative rates conditioned on $R = A$.
They are respectively given by
\begin{equation}
\label{asm:fprate}
	\alpha = \frac{1}{|A^c|} \sum_{x \in A^c} \mathbf{1}_{\tilde{A}}(x)
\end{equation}
and
\begin{equation}
\label{asm:tprate}
	\beta = \frac{1}{|A|} \sum_{x \in A} \mathbf{1}_{\tilde{A}}(x).
\end{equation}



Since the random approximate set is \emph{random}, properties like its \emph{false positive rate} and \emph{false negative rate} are also random, respectively modeled by the random false positive rate $\alpha$ and false negative rate $\beta$.

We denote the \emph{first-order} random approximate set generative model by $\tilde{R}$.
The joint distribution of $\tilde{R}$, $\alpha$, $\beta$, and $R$ given a universal set $U$ has a probability density
\begin{equation}
f(Y, a, b, X | U).
\end{equation}
By the axioms of probability theory, this may be decomposed into
\begin{equation}
f(Y, a, b, X | U) =
f(Y | a, b, X, U)
f(a, b | X)
f(X | U).
\end{equation}
We typically omit the explicit reference to $U$, since it may usually be understood as implicit to the model.

The object of central interest is the distribution of $\tilde{R}$ given $R$.
The conditional distribution of $\tilde{R}$ given $R = X$ is denote by $\tilde{X}$.
By the axioms of probability,
\begin{equation}
f(\tilde{X}, \alpha, \beta) =
f(\tilde{X} | \alpha, \beta )
f(\alpha, \beta | R).
\end{equation}



There are a few natural partitions.







If the rates happen to pick out a specific set in the support, then the result is a degenerate distribution, e.g., $\tilde{A}$ given $\alpha = 0$ and $\beta = 0$ is degenerate where all probability mass is assigned to $A$.

We denote the distributions of $\tilde{X}$ given $\mathbb{E}[\alpha] = \fprate$ and $\tilde{X}$ given $\mathbb{E}[\beta] = \fnrate$ respectively by $\tilde{X}[\fprate][-]$ and $\tilde{X}[+][\fnrate]$.
An object of central interest is the distribution of $\tilde{X}$ given $\mathbb{E}[\alpha] = \fprate$ and $\mathbb{E}[\beta] = \fnrate$, denoted by
\begin{equation}
\tilde{X}[\fprate][\fnrate].
\end{equation}

If we \emph{sample} from $\tilde{A}[\fprate][\fnrate]$, some set $Y \in 2^U$ with false positive rate $a$ and false negative rate $b$ will be realized with probability $f_{\tilde{A}[\fprate][\fnrate] | \alpha, \beta}(Y | a, b)$.
However, as the number of samples goes to infinity, the mean false positive and false negative rates go to $\fprate$ and $\fnrate$ respectively.

Random \emph{positive} and \emph{negative} approximate sets are special cases respectively given by the following definitions.
\begin{definition}
\label{def:pos_approx_set}
A random approximate set $\tilde{A}[0][+]$ is a random \emph{positive} approximate set denoted by $\tilde{A}_+$.
\end{definition}
\begin{definition}
\label{def:neg_approx_set}
A random approximate set $\tilde{A}[-][0]$ is a random \emph{negative} approximate set denoted by $\tilde{A}_-$.
\end{definition}
By these definitions, every realization of $\tilde{A}_+$ and $\tilde{A}_-$ are respectively \emph{supersets} or \emph{subsets} of $A$.
The complement of a random positive (negative) approximate set is a random negative (positive) approximate set.

By \cref{asm:fpr_fnr_r_indep} and by the axioms of probability,
\begin{equation}
f(\tilde{X}, \alpha, \beta) =
f(\tilde{X} | \alpha, \beta)
f(\alpha | R) f(\beta | R).
\end{equation}

Every statistical property of the first-order random approximate set model is entailed by \cref{asm:fprate,asm:tprate}.
Furthermore, these assumptions generally hold in practice, e.g., the Bloom filter\cite{bf} and Perfect hash filter\cite{phf} are two separate implementations of the random positive approximate set in which these assumptions hold.

Suppose the first-order random approximate sets are over the universal set $U$.
Compositions of first-order random approximate sets over the Boolean algebra $(2^U,\cup,\SetIntersection,\SetComplement,\EmptySet,U)$, or random approximate sets of random approximate sets, are not closed over the \emph{first-order} model.
These compositions are addressed in the higher-order model below.

\subsection{Higher-order model}
\label{sec:higher_order_model}

Composing \emph{random approximate sets} over the Boolean algebra $(\Sigma,\SetUnion,\SetIntersection,\SetComplement,\EmptySet,\Set{U})$, where $\Sigma \subseteq \PS{\Set{U}}$ since, for instance, if a \emph{deterministic} algorithm implements the model some elements in $\PS{\Set{U}}$ may not be reachable.
As a result, to satisfy the identity and complementation axioms required by Boolean algebras, we make $\EmptySet$ and $\Set{U}$ available in the model as special cases.
\begin{remark}
	Alternatively, these axioms may be satisfied by making the empty set and the universal set \emph{degenerate} cases, i.e., $\Prob{\ASet{\EmptySet} = \EmptySet} = 1$ and $\Prob{\ASet{U} = \Set{U}} = 1$.
\end{remark}

Furthermore, we may replace any of the operators in the Boolean algebra with \emph{random approximations} that model the noisy or rate-distorted channel previously described, i.e., these operators may themselves be constructors for random approximate sets, e.g., $\Set{A} \, \AT{\SetUnion}[\fprate][\tprate] \, \Set{B} \sim \AT{(\SetUnion[\Set{A}][\Set{B}])}[\tprate][\fprate]$ where $\AT{\SetUnion}[\fprate][\tprate]$ maps negatives to positives with probability $\fprate$ and maps positives to negatives with probability $\tprate$.

A natural mapping is provided by the \emph{identity} function $\Fun{id} \colon \PS{\Set{X}} \mapsto \PS{\Set{X}}$, which is defined as
\begin{equation}
\Fun{id}(\Set{A}) \coloneqq \Set{A}\,.
\end{equation}
However, suppose we only have an \emph{approximation} of the identity function, denoted by $\APFun{id}[\fprate][\tprate]$, such that $\APFun{id}[\fprate][\tprate](\Set{A}) \sim \ASet{A}[\fprate][\tprate]$.
Then $\APFun{id}[\fprate][\tprate]$ generates sets consistent with the random approximate set model.

If we compose random approximate sets, then we have \emph{higher-order} random approximate sets.
\begin{theorem}
\label{thm:composition_rates}
	The composition of random approximate identity functions $\APFun{id}[\fprate][\tprate] \circ \APFun{id}[\fprate'][\tprate']$ generates random approximate sets with a true positive rate $\tprate \tprate' + \fnrate \fprate'$ and false positive rate $\fprate \tprate' + \tnrate \fprate'$.
\end{theorem}
\begin{proof}
	Let $x \in A$. In the inner approximation $\APFun{id}[\fprate'][\tprate']$, element $x$ tests positive with probability $\tprate'$ and negative with probability $\fnrate' = 1 - \tprate'$.
	In the outer approximation $\APFun{id}[\fprate][\tprate]$, a positive is retained with probability $\tprate$ and a negative is promoted to positive with probability $\fprate$.
	By the law of total probability, the composed true positive rate is
	$\tprate' \cdot \tprate + \fnrate' \cdot \fprate = \tprate \tprate' + \fnrate \fprate'$.
	Similarly, let $x \notin A$. In the inner approximation, $x$ tests positive with probability $\fprate'$ and negative with probability $\tnrate' = 1 - \fprate'$.
	The composed false positive rate is
	$\fprate' \cdot \tprate + \tnrate' \cdot \fprate = \fprate \tprate' + \tnrate \fprate'$.
\end{proof}

\begin{definition}
	The \emph{iterated} function $\Fun{f}^k$ is defined as $k$ compositions of $\Fun{f}$ where $\Fun{f}^0$ denotes the (non-random) \emph{identity} function.
\end{definition}
The composition $\left(\APFun{id}[\fprate][\tprate]\right)^k$ generates $k$-th order random approximate sets where the \emph{zero-th} order random approximation is the identity, i.e., $\left(\APFun{id}\right)^0 = \Fun{id}$.

The function being approximated may take other forms, like \emph{set-complementation} or \emph{set-union}, e.g.,
let $\SetUnion \colon \PS{\Set{X}} \times \PS{\Set{X}} \mapsto \PS{\Set{X}}$ be approximated by $\APFun{\SetUnion}[\fprate][\tprate] \colon \PS{\Set{X}} \times \PS{\Set{X}} \mapsto \PS{\Set{X}}$.
Then, $\Set{A} \APFun{\SetUnion}[\fprate][\tprate] \Set{B}$ is a random approximate set of $\SetUnion[\Set{A}][\Set{B}]$ as before.
However, $\APFun{\SetUnion}[\fprate'][\tprate'] \circ \APFun{id}[\fprate][\tprate]$ generates second-order random approximate sets.

\begin{theorem}
\label{thm:second_order}
	Given a random approximate set $\ASet{A}[\fprate_1][\tprate_1]$, a random approximation of $\ASet{A}[\fprate_1][\tprate_1]$ with a false positive rate $\fprate_2$ and true positive rate $\tprate_2$ is a \emph{second-order} random approximate set of $\Set{A}$ with a false positive rate $\fprate = \fprate_1 \tprate_2 + \tnrate_1 \fprate_2$ and true positive rate $\tprate = \tprate_1 \tprate_2 + \fnrate_1 \fprate_2$, denoted by $\Set{A}^{\sigma^2}(\tprate,\fprate)$.

	This result may be recursively applied to derive arbitrary $k$-th order random approximate sets as given by
	\begin{equation}
		\Set{A}^{\sigma^k} = \left(\Set{A}^{\sigma^{k-1}}\right)^{\sigma}
	\end{equation}
	where the \emph{zero-th} order $\Set{A}^{\sigma^0} = \Set{A}$.
\end{theorem}

In an \emph{algebra of sets} $(\PS{\Set{U}}, \SetIntersection, \SetUnion, \SetComplement, \Set{U}, \EmptySet)$, we may compose sets to form new sets.
When these sets model \emph{random approximate sets}, then their compositions model \emph{higher-order} random approximate sets, e.g.,
$\SetUnion[\ASet{A}[\fprate_1][\tprate_1]][\ASet{B}[\fprate_2][\tprate_2]]$
models a higher-order random approximate set which does not obey the first-order model; rather, it partitions the negative set such that each partition may have a different false positive rate and similarly for the positive set.

This contrasts with $\AT{\left(\SetUnion[\Set{A}][\Set{B}]\right)}[\fprate][\tprate]$, in which the false positive rate for the negative elements are uniformly distributed and likewise for the positive elements.
We call these random approximate sets \emph{first-order} approximations.
For completeness, the \emph{zeroth-order} are the objective sets, e.g., the zeroth-order approximation of $\SetUnion[\Set{A}][\Set{B}]$ is $\SetUnion[\Set{A}][\Set{B}]$.
The complexity of the probabilistic model increases as the order increases.

\subsection{Probability space}
\label{sec:prob_model}
Suppose the universal set is $U$ and we have some process that generates approximations of some objective set $A$ that is compatible with the axioms of the random approximate set model.

The process generates subsets of $U$, or alternatively, the \emph{sample space} is $\Sigma = 2^U$.
A primary objective in \emph{probability modeling} is assigning \emph{probabilities} to \emph{events}.
Suppose we have some \emph{probability function} $\ProbFn : \Sigma \mapsto [0,1]$.
The \emph{probability} of some event $A \in \Sigma$ is denoted by $\Prob{A}$.


These are the \emph{elementary events} of the probability space.
The random approximate set model given $R = Y$ is given by the \emph{probability space}
\begin{equation}
\left(\Omega = 2^U, \PS{\Omega}, P\right),
\end{equation}
where $\Omega$ is the \emph{sample space}, $\PS{\Omega}$ is the set of all events, and $P : \PS{\Omega} \mapsto [0,1]$ is the probability set function.


%The \emph{probability space} of random approximate sets given an 
%objective set
%$\Set{Y} \subset \Set{U}$ is given by the triple
%\begin{equation}%
%	\left(\vec{1}, \{0,1\}^u, \Prob{\;\cdot | \vec{y}}\right)
%\end{equation}
%The relative frequency of any event $\vec{x}$ in $\{0,1\}^u$ converges to $\Prob{\AVec{X} = \vec{x} | \OVec{y}}$ as the number of times the random approximate set of $\OVec{y}$ is generated goes to infinity.

%%By \cref{def:bijection}, we use the Boolean algebras
%%$\left(
%%\PS{\Set{U}},\SetIntersection,\cup,\SetComplement,\EmptySet,\Set{U}
%%\right)$ and $(\{0,1\}^u,\land,\lor,\neg,\vec{0},\vec{1})$ interchangeably.

Consider an objective set $A$ and a random approximate set  and suppose we are uncertain about which elements are their respective members.
We model the uncertainty of the elements of $A$ by the Boolean random vector $\vec{A} = \Tuple{\RV{A_1},\ldots,\RV{A_u}}$ where $\RV{A_j} = \SetIndicator{A}\left(x_{(j)}\right)$ for $j=1,\ldots,u$.
Similarly, we model the uncertainty of the elements of the $\tilde{A}$ by $\tilde{\vec{A}}^\fprate_\tprate = \Tuple{\tilde{A}_1,\ldots,\tilde{A}_u}$.

The joint probability that $\tilde{\vec{A}}^\fprate_\tprate = \vec{x}$ and $\vec{A} = \vec{y}$ is denoted by $\Prob{\tilde{\vec{A}}^\fprate_\tprate = \vec{x},\vec{A} = \vec{y}}$.
By the axioms of probability, the joint probability may be rewritten as
\begin{equation}
    \Prob{\tilde{\vec{A}}^\fprate_\tprate = \vec{x},\vec{A} = \vec{y}} =
        \Prob{\tilde{\vec{A}}^\fprate_\tprate = \vec{x} | \vec{A} = \vec{y}}
        \Prob{\vec{A} = \vec{y}}.
\end{equation}
By \cref{asm:fprate,asm:tprate}, $\tilde{A}_j$ is only dependent on 
$\RV{A_j}$ for $j=1,\ldots,u$ and thus by the axioms of probability
\begin{equation}
    \Prob{\tilde{\vec{A}}^\fprate_\tprate = \vec{x},\vec{A} = \vec{y}} = 
    \Prob{\vec{A} = \vec{y}} 
        \prod_{j=1}^{u} \Prob{\tilde{A}_j = x_j | \RV{A_j} = y_j}.
\end{equation}
If it is given that $\tilde{\vec{A}}^\fprate_\tprate = \vec{y}$, i.e., the elements in the 
objective set are known, by the axioms of probability the conditional 
probability is
\begin{equation}
    \Prob{\tilde{\vec{A}}^\fprate_\tprate = \vec{x} | \vec{A} = \vec{y}} = \prod_{j=1}^{u} 
    \Prob{\tilde{A}_j = x_j | \RV{A_j} = y_j}
\end{equation}
where $\fprate = \Prob{\tilde{A}_j=1 | \RV{A_j}=0}$ and $\tprate = \Prob{\tilde{A}_j=1 | \RV{A_j}=1}$.


The relative frequency of any event $\vec{x}$ in $\{0,1\}^u$ converges to $\Prob{\tilde{\vec{X}} = \vec{x} | \tilde{\vec{y}}}$ as the number of times the random approximate set of $\tilde{\vec{y}}$ is generated goes to infinity.

%The relative frequency of any event $\vec{x}$ in $\{0,1\}^u$ converges to 
%$\Prob{\tilde{\vec{X}} = \vec{x} | \OVec{y}}$ as the number of times the 
%algorithm is applied to the objective set $\OVec{y}$ is repeated goes to 
%infinity.

%The sample space (the set of mutually exclusive outcomes
%that may be observed) is given by some subset of $\PS{\Set{U}}$. Sets of 
%outcomes are denoted events where an event that contains one outcome is 
%elementary event and the event containing all outcomes is the sample space 
%$\PS{\Set{U}}$.

%Suppose we an objective set $\Set{S}$ and $n$ approximations drawn from
%$\ASet{S}(\fprate,\tprate)$, denoted by $\ASet{S}(\fprate = \fprateob_j,
%\tprate = \tprateob_j)$ for $j=1,\ldots,n$.

%The \emph{distribution} of false positive rates 
%$\fprateob_1,\ldots,\fprateob_n$ for a sample of $n$ approximate sets 
%$\ASet{S}[1],\ldots,\ASet{S}[n]$, where each is said to have an 
%\emph{expected} false positive rate $\fprate$, has a central tendency 
%around $\fprate$. The same is true for the true positive rate 
%$\tprate$.


Consider the following example.
\begin{example}
	Suppose the universal set is $\{ x_1,x_2 \}$ and consider the distribution of the Bernoulli set $\tilde{\{x_1\}}^\fprate_\fnrate$. The probability mass function $p_{\tilde{\{x_1\}}^\fprate_\fnrate}$ is given by
	\begin{equation}
	p_{\tilde{\{x_1\}}^\fprate_\fnrate}(\Set{X}) =
	\begin{cases} 
	\fnrate (1-\fprate) & \Set{X} = \EmptySet,\\
	\fnrate \fprate     & \Set{X} = \{x_2\},\\
	(1-\fnrate)(1-\fprate)     & \Set{X} = \{x_1\},\\
	(1-\fnrate)\fprate         & \Set{X} = \{x_1,x_2\}.
	\end{cases}
	\end{equation}
\end{example}

The true set, call it $\Set{A}$, being modeled is latent, and we observe a sample from the Bernoulli set $\tilde{\Set{A}}$, i.e., $\tilde{\Set{A}}$ is a random set. As sets, most operations we can perform on $\Set{A}$ are also valid on $\tilde{\Set{A}}$, but operations on $\tilde{\Set{A}}$ will not necessarily yield the same results as operations on $\Set{A}$. We call this the approximation error, and different operations will yield different approximation errors. For example, $\Pr\{\Set{A} = \tilde{\Set{A}}\}$ is the probability that the approximation $\tilde{\Set{A}}$ is equal to the true set $\Set{A}$. This is a function of the Bernoulli model, and in theory each element in $\Set{U}$ may have a different error rate. We denote the number of parameters required to specify the error rate of the Bernoulli model as the \emph{order} of the Bernoulli model.

%%%%%%%%%%%%%%%%%%%%%%%%%%%%%%%%%

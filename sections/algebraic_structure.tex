%%%%%%%%%%%%%%%%%%%%%%%%%%%%%%%%%%
% Algebraic structure
%%%%%%%%%%%%%%%%%%%%%%%%%%%%%%%%%%

\section{Algebraic structure}
\label{sec:algebraic_structure}
We consider the algebraic structure of approximate sets under set-theoretic operations.
In particular, we examine whether the collection of approximate sets forms a semigroup or monoid under union and intersection.

\begin{definition}
A \emph{semigroup} is a set $S$ equipped with an associative binary operation $\cdot \colon S \times S \to S$.
\end{definition}

\begin{definition}
A \emph{monoid} is a semigroup $(S, \cdot)$ with an identity element $e \in S$ satisfying $e \cdot x = x \cdot e = x$ for all $x \in S$.
\end{definition}

\subsection{Semigroup under union and intersection}
The collection of approximate sets with arbitrary false positive and false negative rates forms a semigroup under both union and intersection.
By the results of \cref{sec:set_theory}, the union or intersection of two approximate sets is also an approximate set, with derived error rates that depend on the structural properties of the operands.
Since union and intersection are associative operations on $2^U$, they are also associative on the subcollection of approximate sets, and the semigroup axiom is satisfied.

However, the collection of approximate sets with \emph{fixed} expected false positive and false negative rates $\fprate$ and $\fnrate$ is \emph{not} closed under set-theoretic operations.
By \cref{sec:set_theory}, the output error rates of a union or intersection depend on the structural properties (e.g., the relative sizes $\alpha_1, \alpha_2$) of the input sets.
Consequently, such a restricted collection does not form a semigroup.

\subsection{Obstruction to monoid structure}
As specified, the semigroup of approximate sets with arbitrary error rates is not a monoid.
To form a monoid under union, the identity element $\EmptySet$ must be a member of the collection.
The value constructor $\operatorname{f} \colon 2^U \to 2^U$ is, in general, non-surjective (see \cref{sec:adt}), so there may be no objective set $A$ such that $\operatorname{f}(A) = \EmptySet$.
Furthermore, even if $\EmptySet \in \Image(\operatorname{f})$, different implementations of approximate set generators may produce different representations of the empty set.
An analogous obstruction applies to the intersection identity $\Set{U}$.

\subsection{Degenerate cases and the monoid fix}
If a monoid is desired, we introduce the following degenerate cases as axioms of the value constructor:
\begin{enumerate}
    \item $\operatorname{f}(\EmptySet) = \EmptySet$, i.e., the probabilistic model is degenerate with respect to the empty set, so that $\ASet{\EmptySet} = \EmptySet$ with probability $1$.
    \item $\operatorname{f}(\Set{U}) = \Set{U}$, i.e., the probabilistic model is degenerate with respect to the universal set, so that $\ASet{U} = \Set{U}$ with probability $1$.
\end{enumerate}
Since $\SetComplement[\EmptySet] \equiv \Set{U}$, specifying one degenerate case implies the other under complementation.

With these degenerate cases, the collection of approximate sets over $2^U$ forms:
\begin{itemize}
    \item a \emph{commutative monoid} under union, with identity element $\EmptySet$, and
    \item a \emph{commutative monoid} under intersection, with identity element $\Set{U}$.
\end{itemize}

\begin{remark}[Generic programming]
As monoids, approximate sets may be used with generic programming algorithms that impose monoidal type constraints.
A canonical example is the \emph{reduce} (fold) operator.
Given a collection of approximate sets $\ASet{A}_1, \ldots, \ASet{A}_n$, the reduction
\begin{equation}
    \bigoplus_{i=1}^{n} \ASet{A}_i
\end{equation}
requires an identity element and an associative binary operation --- precisely the monoid structure established above.
For instance, $\bigoplus = \cup$ with identity $\EmptySet$ computes the $n$-ary union of approximate sets, a construction that arises naturally in approximate Boolean search (see \cref{sec:adt}).
\end{remark}

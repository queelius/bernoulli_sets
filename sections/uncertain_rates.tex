%%%%%%%%%%%%%%%%%%%%%%%%%%%%%%%%
% Interval -- uncertainty
%%%%%%%%%%%%%%%%%%%%%%%%%%%%%%%%

\section{Interval arithmetic for uncertain rates}
\label{sec:intervals}
In practice, the expected false positive and true positive rates may not be known exactly.
This section develops an \emph{interval arithmetic} approach: each uncertain rate is represented by an interval guaranteed to contain the true value, and the error-rate formulas of \cref{sec:set_theory} are evaluated over these intervals to produce guaranteed bounds on the output rates.

\begin{definition}
An interval is a convex set of real numbers.
We denote by $\Interval{\underbar{x},\bar{x}}$ an interval with a lower-bound $\underbar{x}$ and an upper-bound $\bar{x}$.
\end{definition}

A confidence interval may be specified in interval notation.
Here, however, we consider an algebra for interval arithmetic and put it to use quantifying our ignorance about the distribution of parameters after, for instance, a union operation.

The error-rate formulas of \cref{sec:set_theory} depend upon the false positive and false negative rates being \emph{known}.
Any parameters that are not known with certainty may be replaced by intervals that (are assumed to) contain the true value.
As a consequence, the output error rate will also be an interval.

\emph{Maximum} uncertainty is when the parameter value is in the interval $[0,1]$, e.g., $[\fprate] = [0,1]$, and \emph{minimum} uncertainty is when the parameter is some value in the degenerate interval $[x,x]$, e.g., $[\fprate] = [.2,.2]$. The more certain---the smaller the width of the intervals---the more certain the output rate.

When using interval arithmetic, the \emph{dependency problem} can lead to overly pessimistic bounds.
In our case, the formulae are simple enough to ensure dependencies are satisfied. The following example illustrates the approach for the union false positive rate.

\begin{example}
	Suppose two approximate sets have false positive rates known only to lie in $[\fprate_1]$ and $[\fprate_2]$ respectively.
	The union FPR formula (\cref{thm:union_fprate}) is $\fprate = 1 - (1-\fprate_1)(1-\fprate_2)$, which is monotonically increasing in both $\fprate_1$ and $\fprate_2$.
	Evaluating at the endpoints yields guaranteed bounds:
	\begin{equation}
		[\fprate] = \bigl[1 - (1 - \underline{\fprate}_1)(1 - \underline{\fprate}_2),\;
		             1 - (1 - \overline{\fprate}_1)(1 - \overline{\fprate}_2)\bigr].
	\end{equation}
	For instance, if $[\fprate_1] = [0.01, 0.05]$ and $[\fprate_2] = [0.02, 0.08]$, then $[\fprate] = [0.0298, 0.126]$.
\end{example}

We may not be interested in the \emph{expected} value, but the smallest set of values such that with probability $1-\gamma$ the true rate
realizes some value in the set, which is typically an \emph{interval}, i.e., a confidence interval.

Intervals represent an uncertainty and they manifest themselves in two independent ways.
The common notion of the \emph{confidence interval} is a product of the probabilistic model, i.e., the realized true positive rate $\tprateob$, which is normally centered around the expected true positive rate $\tprate$ as discussed in \cref{sec:asymtotic}.
We may use \emph{interval arithmetic} and replace point values with interval values, point values being a degenerate case.
Basic interval arithmetic is presented in~\cite{basicinterval}; the foundational treatment is due to Moore~\cite{mooreInterval}.

A set sampled from $\tilde{A}[\fprate][\tprate]$ is an approximate set such that the $\gamma\cdot 100\%$ asymptotic confidence intervals for the realized false negative and false positive rates are respectively given by
\begin{equation}
\Interval{\fnrate} = \fnrate \pm \sqrt{\frac{\fnrate(1-\fnrate)}{p}}\, \Phi^{-1}\!\left(\frac{1+\gamma}{2}\right)
\end{equation}
and
\begin{equation}
\Interval{\fprate} = \fprate \pm \sqrt{\frac{\fprate(1-\fprate)}{n}}\, \Phi^{-1}\!\left(\frac{1+\gamma}{2}\right),
\end{equation}
where $p$ and $n$ are the number of positives and negatives respectively and $\Phi^{-1}$ is the inverse CDF of the standard normal (see \cref{eq:conf_fpr,eq:conf_tpr}).

\subsection{Interval propagation through set operations}

The simple error-rate formulas---union FPR, intersection FNR, and complement---are monotonically increasing in each input rate.
The weighted formulas (union FNR, intersection FPR, and set difference rates) have \emph{mixed monotonicity}: they increase in same-type rates ($\fnrate$ in $\fnrate$, $\fprate$ in $\fprate$) and decrease in cross-type rates ($\fnrate$ in $\fprate$, $\fprate$ in $\fnrate$).
Replacing point-valued rates with intervals and evaluating at the appropriate endpoints for each variable yields guaranteed bounds on the output rates.

\begin{theorem}[Complement]
\label{thm:uncertain_rates_comp_set}
The \emph{complement} of an approximate set with a false negative rate $\Interval{\fnrate}$ and false positive rate $\Interval{\fprate}$ is an approximate set with a false negative rate $\Interval{\fprate}$ and false positive rate $\Interval{\fnrate}$.
\end{theorem}
\begin{proof}
Follows immediately from \cref{thm:complement_rates}: complementation swaps positives and negatives, so it swaps the false positive and false negative rates. Since the interval endpoints swap accordingly, the result holds.
\end{proof}

\begin{theorem}[Union]
\label{thm:uncertain_rates_union}
Given two approximate sets with interval-valued rates $[\fprate_1]$, $[\fnrate_1]$ and $[\fprate_2]$, $[\fnrate_2]$, the union has interval-valued false positive rate
\begin{equation}
[\fprate] = \bigl[1 - (1 - \underline{\fprate}_1)(1 - \underline{\fprate}_2),\;
             1 - (1 - \overline{\fprate}_1)(1 - \overline{\fprate}_2)\bigr],
\end{equation}
since the union FPR formula $\fprate = 1 - (1-\fprate_1)(1-\fprate_2)$ from \cref{thm:union_fprate} is monotonically increasing in both $\fprate_1$ and $\fprate_2$.
Since the union FNR (\cref{eq:union_fnr}) increases in $\fnrate_1, \fnrate_2$ and decreases in $\fprate_1, \fprate_2$, the upper bound is attained at $(\overline{\fnrate}_1, \overline{\fnrate}_2, \underline{\fprate}_1, \underline{\fprate}_2)$ and the lower bound at $(\underline{\fnrate}_1, \underline{\fnrate}_2, \overline{\fprate}_1, \overline{\fprate}_2)$.
\end{theorem}

\begin{remark}
Intersection and set difference follow analogously: the intersection FNR has the same form as the union FPR (by the De Morgan duality noted in \cref{sec:set_ops_summary}), and the set difference is handled by substituting complemented intervals per \cref{thm:uncertain_rates_comp_set}.
\end{remark}



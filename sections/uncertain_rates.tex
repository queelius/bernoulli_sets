%%%%%%%%%%%%%%%%%%%%%%%%%%%%%%%%
% Interval -- uncertainty
%%%%%%%%%%%%%%%%%%%%%%%%%%%%%%%%

\section{Uncertain rates}
\label{sec:intervals}
In practice, the expected false positive and true positive rates may not be known exactly.
This section develops an \emph{interval arithmetic} approach: each uncertain rate is represented by an interval guaranteed to contain the true value, and the error-rate formulas of \cref{sec:set_theory} are evaluated over these intervals to produce guaranteed bounds on the output rates.

\begin{definition}
An interval is a convex set of real numbers.
We denote by $\Interval{\underbar{x},\bar{x}}$ an interval with a lower-bound $\underbar{x}$ and an upper-bound $\bar{x}$.
\end{definition}

A confidence interval may be specified in interval notation.
Here, however, we consider an algebra for interval arithmetic and put it to use quantifying our ignorance about the distribution of parameters after, for instance, a union operation.

The performance measures summarized by \cref{tbl:perf_sum} depend upon the false positive rate $\fprate$, false negative rate $\fnrate$, and proportion of positives $\lambda$ being \emph{known}.
Any parameters that are not known with certainty may be replaced in the above table by intervals that (are assumed to) contain the expected value.
As a consequence, the performance measure will also be an interval.

\emph{Maximum} uncertainty is when the parameter value is in the interval $[0,1]$, e.g., $[\lambda] = [0,1]$, and \emph{minimum} uncertainty is when the parameter is some value in the degenerate interval $[x,x]$, e.g., $[\fprate] = [.2,.2]$. The more certain--the smaller the width of the intervals--the more certain the performance measure.

When using interval arithmetic, the \emph{dependency problem} can lead to overly pessimistic bounds.
In our case, the formulae are simple enough to ensure dependencies are satisfied. We show the results of an uncertain proportion of positives $[\lambda]$ for the \emph{accuracy} measure in the following example.

\begin{example}
	Suppose we wish to determine the expected accuracy given that the proportion of positives $\lambda$ is known to be some value in the interval $[\lambda]$, and that the false positive and false negative rates lie in $[\fprate]$ and $[\fnrate]$ respectively.
	By the accuracy formula $\acc = \lambda(1-\fnrate) + (1-\lambda)(1-\fprate)$, the expected accuracy is some value in the interval
	\begin{equation}
		\acc([\fprate],[\fnrate] ; [\lambda]) =
		\biggl[
		\overline{\lambda}(1-\overline{\fnrate}) +
		(1-\overline{\lambda})(1-\overline{\fprate}),\;
		\underline{\lambda}(1-\underline{\fnrate}) +
		(1-\underline{\lambda})(1-\underline{\fprate})
		\biggr],
	\end{equation}
	where the lower bound is attained at the worst-case rates $(\overline{\fprate}, \overline{\fnrate}, \overline{\lambda})$ and the upper bound at the best-case rates $(\underline{\fprate}, \underline{\fnrate}, \underline{\lambda})$.
\end{example}

We may not be interested in the \emph{expected} value, but the smallest set of values such that with probability $1-\gamma$ the true rate
realizes some value in the set, which is typically an \emph{interval}, i.e., a confidence interval.

Intervals represent an uncertainty and they manifest themselves in two independent ways.
The common notion of the \emph{confidence interval} is a product of the probabilistic model, i.e., the realized true positive rate $\tprateob$, which is normally centered around the expected true positive rate $\tprate$ as discussed in \cref{sec:asymtotic}.
We may use \emph{interval arithmetic} and replace point values with interval values, point values being a degenerate case.
Basic interval arithmetic is presented in \cite{basicinterval}.

A set sampled from $\tilde{A}[\fprate][\tprate]$ is an approximate set such that the $(1-\gamma)\cdot 100\%$ asymptotic confidence intervals for the realized false negative and false positive rates are respectively given by
\begin{equation}
\Interval{\fnrate} = \fnrate \pm \sqrt{\frac{\fnrate(1-\fnrate)}{p}}\, \Phi^{-1}\!\left(\frac{\gamma}{2}\right)
\end{equation}
and
\begin{equation}
\Interval{\fprate} = \fprate \pm \sqrt{\frac{\fprate(1-\fprate)}{n}}\, \Phi^{-1}\!\left(\frac{\gamma}{2}\right),
\end{equation}
where $p$ and $n$ are the number of positives and negatives respectively and $\Phi^{-1}$ is the inverse CDF of the standard normal (see \cref{eq:conf_fpr,eq:conf_tpr}).

\subsection{Interval propagation through set operations}

The error-rate formulas of \cref{sec:set_theory} are monotone in the input rates, so replacing point-valued rates with intervals and evaluating at the appropriate endpoints yields guaranteed bounds on the output rates.

\begin{theorem}[Complement]
\label{thm:uncertain_rates_comp_set}
The \emph{complement} of an approximate set with a false negative rate $\Interval{\fnrate}$ and false positive rate $\Interval{\fprate}$ is an approximate set with a false negative rate $\Interval{\fprate}$ and false positive rate $\Interval{\fnrate}$.
\end{theorem}
\begin{proof}
Follows immediately from \cref{thm:complement_rates}: complementation swaps positives and negatives, so it swaps the false positive and false negative rates. Since the interval endpoints swap accordingly, the result holds.
\end{proof}

\begin{theorem}[Union]
\label{thm:uncertain_rates_union}
Given two approximate sets with interval-valued rates $[\fprate_1]$, $[\fnrate_1]$ and $[\fprate_2]$, $[\fnrate_2]$, the union has interval-valued false positive rate
\begin{equation}
[\fprate] = \bigl[1 - (1 - \underline{\fprate}_1)(1 - \underline{\fprate}_2),\;
             1 - (1 - \overline{\fprate}_1)(1 - \overline{\fprate}_2)\bigr],
\end{equation}
since the union FPR formula $\fprate = 1 - (1-\fprate_1)(1-\fprate_2)$ from \cref{thm:union_fprate} is monotonically increasing in both $\fprate_1$ and $\fprate_2$.
The false negative rate of the union is bounded by evaluating \cref{eq:union_fnr} at the worst-case and best-case endpoints of $[\fnrate_1]$, $[\fnrate_2]$, $[\fprate_1]$, and $[\fprate_2]$ respectively.
\end{theorem}

\begin{remark}
Intersection and set difference follow analogously: the intersection FNR has the same form as the union FPR (by the De Morgan duality noted in \cref{sec:set_ops_summary}), and the set difference is handled by substituting complemented intervals per \cref{thm:uncertain_rates_comp_set}.
\end{remark}

\begin{example}[Interval PPV]
Given interval-valued rates, the positive predictive value inherits interval bounds.
Since $\ppv \approx \overline{t}/(\overline{t} + \overline{f})$ for large sets (see \cref{thm:approx_expected_precision}), substituting $\overline{t} = p\,\tprate$ and $\overline{f} = n\,\fprate$ and evaluating at the extremes of $[\fprate]$ and $[\tprate]$ yields
\begin{equation}
[\ppv] \approx \biggl[\frac{p\,\underline{\tprate}}{p\,\underline{\tprate} + n\,\overline{\fprate}},\; \frac{p\,\overline{\tprate}}{p\,\overline{\tprate} + n\,\underline{\fprate}}\biggr],
\end{equation}
providing guaranteed bounds on the expected precision under uncertain rates.
\end{example}


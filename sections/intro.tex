%%%%%%%%%%%%%%%%%%%%%%%%%%%%%%%%
% Introduction
%%%%%%%%%%%%%%%%%%%%%%%%%%%%%%%%

\section{Introduction}

Many computational systems rely on \emph{approximate set representations}---data structures whose membership queries may err, returning false positives or false negatives with known rates.
Bloom filters~\cite{bf}, perfect hash filters~\cite{phf}, and related probabilistic data structures achieve dramatic space savings by tolerating such errors, and individual structures are well understood.
However, practical systems rarely use a single approximate set in isolation.
Encrypted search indexes, database query planners, and distributed systems routinely compose approximate sets through union, intersection, complement, and difference, and the resulting error behavior has traditionally been analyzed on a case-by-case basis.

\paragraph{The gap.}
The existing literature lacks a \emph{compositional algebra} for approximate sets: a framework in which the error rates of any set-theoretic expression are mechanically computable from the error rates of its operands.
Without such a framework, each new combination of approximate sets requires a bespoke analysis, and higher-order compositions---approximate sets of approximate sets---remain largely uncharacterized.

\paragraph{Contribution.}
We introduce the \emph{Bernoulli set model}, a probabilistic framework for \emph{random approximate sets} built on two axioms: element-wise independence of errors and conditional independence of block error rates.
The central result is \emph{compositional closure}: every set-theoretic operation on Bernoulli sets produces a Bernoulli set whose error rates are closed-form functions of the input rates and set cardinalities.
This closure extends to higher-order compositions via a recursive application of the composition theorem.
From these axioms we derive:
\begin{enumerate}[nosep]
    \item the probability distributions of all standard binary classification measures (precision, recall, accuracy, etc.),
    \item closed-form error rates for complement, union, intersection, and set difference,
    \item the joint entropy of error counts,
    \item conditions under which approximate sets form commutative monoids, and
    \item an application to encrypted Boolean search in which compositionality yields all performance guarantees.
\end{enumerate}
The framework is formulated as an abstract data type: any implementation whose membership queries satisfy the Bernoulli axioms---including Bloom filters, perfect hash filters, and their compositions---inherits the full theory automatically.

\paragraph{Related work.}
The original Bloom filter~\cite{bf} introduced approximate set membership with one-sided error.
Subsequent work has produced a rich family of probabilistic data structures---including cuckoo filters~\cite{cuckooFilter} and many variants surveyed by Broder and Mitzenmacher~\cite{broderMitzenmacher}---with improved space efficiency or additional functionality.
However, analyses typically treat each structure in isolation, deriving error rates for a single filter rather than for compositions of filters.
General probabilistic methods for randomized algorithms~\cite{mitzenmacherUpfal} provide the analytical toolkit (concentration inequalities, entropy bounds) but do not address the compositional structure of approximate sets.
The interval arithmetic approach to uncertain parameters~\cite{basicinterval} addresses a related but distinct problem: propagating uncertainty through arithmetic expressions rather than set-theoretic ones.
Our framework complements these lines of work by providing an algebraic layer that sits above any particular implementation.

\paragraph{Organization.}
\Cref{sec:setalgebra} reviews the algebra of sets.
\Cref{sec:bernoulli_model} introduces the Bernoulli set model and its axioms, including the higher-order composition theorem.
\Cref{sec:characteristics} derives the probability distributions of fundamental binary classification measures.
\Cref{sec:set_theory} derives error rates for all four set-theoretic operations and establishes closure properties.
\Cref{sec:relational} treats relational predicates (subset, equality) on Bernoulli sets.
\Cref{sec:intervals} addresses uncertain rate parameters via interval arithmetic.
\Cref{sec:entropy} derives the joint entropy of error counts and establishes space complexity bounds.
\Cref{sec:algebraic_structure} examines the algebraic structure (commutative monoids) of approximate sets.
\Cref{sec:bool_search} demonstrates the framework through an application to encrypted Boolean search.
